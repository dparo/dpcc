% You should not modify anything from here ... -------------
\documentclass[a4paper]{article}
\usepackage[english]{babel}
\usepackage{microtype,etex,listings,color,parskip,hyperref}
\usepackage[margin=2cm]{geometry}
\usepackage{hyperref}

\lstset{
    language=C,
    tabsize=2,
    showstringspaces=false,
    breaklines=true,
    basicstyle=\ttfamily,
    keywordstyle=\color[rgb]{0.1,0.3,0.7}\ttfamily,
    stringstyle=\color[rgb]{0.7,0.1,0.3}\ttfamily,
    commentstyle=\color[rgb]{0.3,0.4,0.3}\ttfamily,
    columns=fixed,
    numberstyle=\sffamily\scriptsize,
    backgroundcolor=\color[rgb]{0.95,0.95,0.95},
    frame=lines,
    framexleftmargin=5pt,
    numbers = left,
    numberstyle = \footnotesize,
}

\lstdefinelanguage{DPCC}{
    keywords={ print, int, float, bool, var, typeof, new, true, false, catch,
        function, return, null, catch, switch, let, var, if, of, in,
    for, while, do, else, case, break },
    keywordstyle=\color{blue}\bfseries,
    ndkeywords={class, export, boolean, throw, implements, import, this},
    ndkeywordstyle=\color{darkgray}\bfseries,
    identifierstyle=\color{black},
    sensitive=false,
    comment=[l]{//},
    morecomment=[s]{/*}{*/},
    commentstyle=\color{purple}\ttfamily,
    stringstyle=\color{red}\ttfamily,
    morestring=[b]',
    morestring=[b]"
}

\lstset{
    language=DPCC,
    extendedchars=true,
    basicstyle=\footnotesize\ttfamily,
    showstringspaces=false,
    showspaces=false,
    numbers=left,
    columns=flexible,
    numberstyle=\tiny,
    numbersep=9pt,
    tabsize=2,
    breaklines=true,
    showtabs=false,
    captionpos=b,
}

\begin{document}

\title{DPCC: DParo's Own C-Alike Compiler}
\author{Davide Paro}
\date{December 2020}

\maketitle


\section*{Project Description}
This project is the implementation of an assignment for a course on \textbf{Compilers}
at the department of Computer Engineering Master Degree Padova (ITA).


The assignment consists in implementing a toy compiler (mostly the frontend side) for a toy language
at our liking.

The assignment specs out the how the compiler should be composed.
We can in fact distinguish these macro components.

\begin{itemize}
\item \textbf{Input Stage} deals with the input byte stream that composes
    the source of the program.
\item \textbf{Lexer/Scanner} has the purpose of grouping characters (lexical analysis)
    together to compose compunded
    structures (called tokens). For the project assignment we can use \textbf{Flex} to aid in
    the code generation for the scanner.
\item \textbf{Parser} for performing the syntax analysis. It is what defines the look \& fell
    (grammar) of the language. For the project assignment we can use \textbf{Bison} to ad in the
    implementation of a good parser.
\item \textbf{Intermediate Code Generator}. The ultimate purpose of a compiler is to produce something
    useful. In this project assignment we are not requested to implement a proper backend. Thus,
    we need to emit a 3AC representation of our input program. More in this later.
\end{itemize}

In particular the final Intermediate representation that we need to emit is based on Three Address Code (3AC), that
is each statement can only have 1 operand at the left hand side of the assignment, and 2 operands at the left hand side
of the assignment, and an operator driving the operation that should be performed.

You can view more about 3AC at the following \href{https://en.wikipedia.org/wiki/Three-address_code}{wikipedia link}.

In practice the emitted 3AC code is on itself a partially valid C program, it's only missing variable
declarations at the top for the temporary variables.


For the specification of the Intermdiate Code that is generated please refer to \hyperref[appendix_a]{appendix A}

So the project requires to produce this kind of 3AC / C hybrid. Control flow is allowed to be implemented
trough the usage of C labels and simple if conditional followed by a goto statement. Inside
the if conditional there can only be a single element composing the expression.

The assignment requires the following features from the programming langugage that we should develop:

\begin{itemize}
    \item Variables declaration, initialization and assignment
    \item Handling of variable scopes. Variable names can be reused if out of scope. Variable
        shadowing may or may not warn/fail/pass depending on the design choices.
    \item Only 2 types of variables: integers, booleans
    \item Assignment statements, print statemnts, if statements, and at least 1 loop statement at will (while, for, \dots)
    \item Handling of simple mathematical expressions that we can encounter in common programming
        languages, addition, subtraction, multiplication, division, modulo, etc \dots
    \item \textbf{Function definition, function calls, and custom user definable types are not required}
\end{itemize}

\clearpage

\subsection*{Intermediate Code Generation: 3AC}
\section*{The prospsoed implementation}
\begin{itemize}
\item \textbf{Input Stage}. The compiler only supports file. The input stream is implemented
    in the following way. The file is loaded and entirely copied into a memory buffer. This is more
    than adequate for what it's necessary to do for the project assignment.
\item Un \textbf{Lexer/Scanner} per la tokenizzazione del sorgente da caratteri a tipi di dati strutturati. La scelta ricade su \textbf{Flex} per la gestione dell'analisi lessicale
\item Un \textbf{Parser} per implementare la sintassi del linguaggio e gestire l'analisi sintattica. La scelta ricade su \textbf{Bison} per la gestione di questa componente
\item Un semplice \textbf{generator di codice intermedio}. Il progetto provede di generare un ibrido Assembly/C/3AC come semplice esempio di gestione di generazione
\end{itemize}



\subsection*{Hello world: More hello world}

\section*{Problem analysis}

...

\section*{Program design}

...

\section*{Evaluation of the program}

...

% \section*{Evaluation of the program} % Optional
%
% ...

\section*{Process description}

...

\section*{Conclusions}

...

\section*{Appendix A: Structure of the Intermediate Code}
\label{appendix_a}

\section*{Appendix: program text}

% Here you should include the program text.
% Do NOT use screenshots or similar methods.
% Below you can see how to use \lstinputlisting{}.



\begin{lstlisting}[language=DPCC]
let a: int[] = 10;
let s = "Hello world";
{

}
\end{lstlisting}

\begin{lstlisting}[language=C]
int main() {

}
char **argv;
\end{lstlisting}


\clearpage
\section*{Appendix B: Example Program Iterative Merge Sort}
\label{appendix_b}

\begin{lstlisting}[language=DPCC]
let array = [
    15, 59, 61, 75, 12, 71,  5, 35, 44,
    6, 98, 17, 81, 56, 53, 31, 20, 11,
    45, 80,  8, 34, 71, 83, 64, 28,  3,
    88, 50, 48, 80,  5
];


for (let curr_size = 1; curr_size < len; curr_size = 2 * curr_size) {
    for (let left_start = 0; left_start < len - 1; left_start = left_start + 2 * curr_size) {
        let mid = len - 1;

        if ((left_start + curr_size - 1) < len - 1) {
            mid = left_start + curr_size - 1;
        }

        let right_end = len - 1;

        if ((left_start + 2 * curr_size - 1) < len - 1) {
            right_end = left_start + 2 * curr_size - 1;
        }

        {
            let l = left_start;
            let m = mid;
            let r = right_end;
            let n1 = m - l + 1;
            let n2 = r - m;

            let L: int[1024];
            let R: int[1024];

            for (let i = 0; i < n1; i++) {
                L[i] = array[l + i];
            }

            for (let i = 0; i < n2; i++) {
                R[i] = array[m + 1 + i];
            }


            let i = 0;
            let j = 0;
            let k = l;

            while (i < n1 && j < n2) {
                if (L[i] <= R[j]) {
                    array[k++] = L[i++];
                } else {
                    array[k++] = R[j++];
                }
            }

            while (i < n1) {
                array[k++] = L[i++];
            }
            while (j < n2) {
                array[k++] = R[j++];
            }
        }
    }
}

print("Sorted array\n");
print(array);
\end{lstlisting}


% \section*{Appendix: test cases} % Optional
%
% ...

% \section*{Appendix: Extensions} % Optional
%
% ...

\end{document}
