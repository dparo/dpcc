% You should not modify anything from here ... -------------
\documentclass[a4paper]{article}
\usepackage[english]{babel}
\usepackage{microtype,etex,listings,color,parskip}
\usepackage[margin=2cm]{geometry}
\usepackage{hyperref}
\lstset{
  language=C,
  tabsize=2,
  showstringspaces=false,
  breaklines=true,
  basicstyle=\ttfamily,
  keywordstyle=\color[rgb]{0.1,0.3,0.7}\ttfamily,
  stringstyle=\color[rgb]{0.7,0.1,0.3}\ttfamily,
  commentstyle=\color[rgb]{0.3,0.4,0.3}\ttfamily,
  columns=fixed,
  numberstyle=\sffamily\scriptsize,
  backgroundcolor=\color[rgb]{0.95,0.95,0.95},
  frame=lines,
  framexleftmargin=5pt,
  numbers = left,
  numberstyle = \footnotesize
}

\lstdefinelanguage{DPCC}{
  keywords={ print, int, float, bool, var, typeof, new, true, false, catch,
             function, return, null, catch, switch, let, var, if, of, in,
             for, while, do, else, case, break },
  keywordstyle=\color{blue}\bfseries,
  ndkeywords={class, export, boolean, throw, implements, import, this},
  ndkeywordstyle=\color{darkgray}\bfseries,
  identifierstyle=\color{black},
  sensitive=false,
  comment=[l]{//},
  morecomment=[s]{/*}{*/},
  commentstyle=\color{purple}\ttfamily,
  stringstyle=\color{red}\ttfamily,
  morestring=[b]',
  morestring=[b]"
}

\lstset{
  language=DPCC,
  extendedchars=true,
  basicstyle=\footnotesize\ttfamily,
  showstringspaces=false,
  showspaces=false,
  numbers=left,
  columns=flexible,
  numberstyle=\tiny,
  numbersep=9pt,
  tabsize=2,
  breaklines=true,
  showtabs=false,
  captionpos=b,
}
% ... until here -------------------------------------------

\begin{document}

% replace X and XXX with the number and title of the assignment:
\title{DPCC: DParo's Own C-Alike Compiler}
% DO NOT ADD YOUR NAME, only your student numbers:
\author{Davide Paro}
\date{December 2020}

\maketitle


%%%%%%%%%%%%%%%%%%%%%%%%%%%%%%%%%%%%%%%%%%%%%%%%%%%%%%%%%%%%%%%%%%%
% NOTE: You MUST read and follow Appendix E of the lecture notes! %
%%%%%%%%%%%%%%%%%%%%%%%%%%%%%%%%%%%%%%%%%%%%%%%%%%%%%%%%%%%%%%%%%%%

\section*{Project Description}
This project is the implementation of an assignment for a course on compilers
of Computer Engineering at Padua.

The project consists in an implmenetation of a toy compiler for a toy language for
which we are free to desing the syntax however we like.


\begin{itemize}
\item \textbf{Lettura del sorgente}. In particolare si può supporre una tradizionale lettura da
  un file.
\item Un \textbf{Lexer/Scanner} per la tokenizzazione del sorgente da caratteri a tipi di dati strutturati. La scelta ricade su \textbf{Flex} per la gestione dell'analisi lessicale
\item Un \textbf{Parser} per implementare la sintassi del linguaggio e gestire l'analisi sintattica. La scelta ricade su \textbf{Bison} per la gestione di questa componente
\item Un semplice \textbf{generator di codice intermedio}. Il progetto provede di generare un ibrido Assembly/C/3AC come semplice esempio di gestione di generazione
\end{itemize}

For the specification of the Intermdiate Code that is generated please refer to \hyperref[appendix_a]{appendix A}

\subsection*{Intermediate Code Generation: 3AC}

\section*{Problem analysis}

...

\section*{Program design}

...

\section*{Evaluation of the program}

...

% \section*{Evaluation of the program} % Optional
%
% ...

\section*{Process description}

...

\section*{Conclusions}

...

\section*{Appendix A: Structure of the Intermediate Code}
\label{appendix_a}

\section*{Appendix: program text}

% Here you should include the program text.
% Do NOT use screenshots or similar methods.
% Below you can see how to use \lstinputlisting{}.



\begin{lstlisting}[language=DPCC]
let a: int[] = 10;
let s = "Hello world";
{

}
\end{lstlisting}

\begin{lstlisting}[language=C]
int main() {

}
char **argv;
\end{lstlisting}


\clearpage
\section*{Appendix B: Example Program Iterative Merge Sort}

\begin{lstlisting}[language=DPCC]
let array = [
    15, 59, 61, 75, 12, 71,  5, 35, 44,
    6, 98, 17, 81, 56, 53, 31, 20, 11,
    45, 80,  8, 34, 71, 83, 64, 28,  3,
    88, 50, 48, 80,  5
];


for (let curr_size = 1; curr_size < len; curr_size = 2 * curr_size) {
    for (let left_start = 0; left_start < len - 1; left_start = left_start + 2 * curr_size) {
        let mid = len - 1;

        if ((left_start + curr_size - 1) < len - 1) {
            mid = left_start + curr_size - 1;
        }

        let right_end = len - 1;

        if ((left_start + 2 * curr_size - 1) < len - 1) {
            right_end = left_start + 2 * curr_size - 1;
        }

        {
            let l = left_start;
            let m = mid;
            let r = right_end;
            let n1 = m - l + 1;
            let n2 = r - m;

            let L: int[1024];
            let R: int[1024];

            for (let i = 0; i < n1; i++) {
                L[i] = array[l + i];
            }

            for (let i = 0; i < n2; i++) {
                R[i] = array[m + 1 + i];
            }


            let i = 0;
            let j = 0;
            let k = l;

            while (i < n1 && j < n2) {
                if (L[i] <= R[j]) {
                    array[k++] = L[i++];
                } else {
                    array[k++] = R[j++];
                }
            }

            while (i < n1) {
                array[k++] = L[i++];
            }
            while (j < n2) {
                array[k++] = R[j++];
            }
        }
    }
}

print("Sorted array\n");
print(array);
\end{lstlisting}


% \section*{Appendix: test cases} % Optional
%
% ...

% \section*{Appendix: Extensions} % Optional
%
% ...

\end{document}
